\documentclass{beamer}

\title{GELI Support for UEFI}
\author{Eric L. McCorkle}

\begin{document}

\begin{frame}
  \titlepage{}
\end{frame}

\section{Disclaimer}

\begin{frame}
  \frametitle{Disclaimer}

  The content of this presentation does not constitute a statement on
  behalf of or represent the position of any company or organization.
  The work described herein was not carried out on behalf of any
  company or organization, including any employer.
\end{frame}

\section{Background}

\begin{frame}
  \frametitle{Background}
  \begin{itemize}
  \item \texttt{geli} is a block-level disk encryption scheme for FreeBSD.
  \item Support for booting in X86 BIOS mode from \texttt{geli}
    volumes added by Allan Jude.
  \item UEFI boot is very different from X86 BIOS.
  \item GELI support for UEFI necessary to support modern hardware,
    UEFI features (secure boot, UEFI variables, etc.)
  \end{itemize}
\end{frame}

\begin{frame}
  \frametitle{UEFI Boot Process}
  \begin{itemize}
  \item UEFI specification provides a number of APIs for device I/O,
    memory allocation, driver registration, etc.
  \item \emph{No} direct access to devices, control over addresses, etc.
  \item Firmware looks for EFI System Partition (ESP), loads boot
    application from standard location
  \item UEFI spec calls for minimum 200Mib ESP, can be larger
  \item FreeBSD UEFI boot process has two steps
  \item \texttt{boot1} is a UEFI application installed to ESP, looks
    for boot partition, loads \texttt{loader.efi}
  \item \texttt{loader.efi} presents the standard FreeBSD boot shell
  \end{itemize}
\end{frame}

\section{Core Issues}

\begin{frame}
  \frametitle{Table of Contents}
  \tableofcontents[currentsection]
\end{frame}

\begin{frame}
  \frametitle{UEFI Benefits and Challenges}
  \begin{itemize}
  \item Benefit: (mostly) free from space constraints, unlike X86 BIOS
  \item Challenge: level of abstraction precludes safely passing
    arbitrary memory from \texttt{boot1} to \texttt{loader.efi}
  \item Challenge: harder to properly implement GELI driver
  \item Challenge: split code base between \texttt{boot1} and
    \texttt{loader.efi}
  \item Benefit: a lot more tools to work with
  \end{itemize}
\end{frame}

\begin{frame}
  \frametitle{Issue: Key Transmission}
  \begin{itemize}
  \item User should only input password once
  \item Na\"{i}ve implementation would require \emph{three} separate times
  \item Need to transmit keys from \texttt{boot1} to
    \texttt{loader.efi}, then to kernel
  \item X86 BIOS pushed password onto \texttt{loader}'s stack, then as
    an environment variable for the kernel
  \item UEFI has stronger separation between stages
  \item Want to support multiple passwords
  \item Hashed passwords better (only incurs one hashing delay)
  \item Ideally, provide straightforward migration to hardware key
    storage mechanisms
  \end{itemize}
\end{frame}

\begin{frame}
  \frametitle{Issue: Split Codebase}
  \begin{itemize}
  \item \texttt{loader.efi} uses \texttt{libstand} API with UEFI backend
  \item \texttt{boot1} used completely separate codebase with its own interface
  \item Codebases were almost completely independent, significant duplication
  \item \texttt{boot1} codebase tended towards minimality, difficult
    to maintain and improve
  \item Any change would require two separate implementations with
    different underlying designs
  \item This code duplication hampered both current as well as planned
    future work
  \end{itemize}
\end{frame}

\begin{frame}
  \frametitle{Issue: GELI Driver Architecture}
  \begin{itemize}
  \item GELI is designed around GEOM, a multi-layered device interface
  \item Can support arbitrarily-complex schemes (GPT/GELIs inside
    GELIs, and so on)
  \item Boot loader support is more limited
  \item \texttt{boot1} would require complete overhaul to support
    GELI-like structures
  \item \texttt{loader.efi} has ability to support ``one-layer''
    schemes (GELIs on partitions)
  \end{itemize}
\end{frame}

\section{Project History}

\begin{frame}
  \frametitle{Table of Contents}
  \tableofcontents[currentsection]
\end{frame}

\begin{frame}
  \frametitle{Timeline of Work}
  \begin{itemize}
  \item Attempt to refactor \texttt{boot1}
  \item First attempt at unifying \texttt{boot1} and
    \texttt{loader.efi} codebases (EFIzation)
  \item First working GELI driver!
  \item Time in code review, use on real hardware
  \item ZFS boot environment issues identified, non-trivial changes to HEAD
  \item Design revision, simplification, establishment of new branches
  \item Key intake buffers go into kernel
  \item Full-disk root-on-ZFS under GELI working on real hardware
  \end{itemize}
\end{frame}

\subsection{Initial Work}

\begin{frame}
  \frametitle{Table of Contents}
  \tableofcontents[currentsection]
\end{frame}

\begin{frame}
  \frametitle{First Refactor of \texttt{boot1}}
  \begin{itemize}
  \item Purpose was to ``rough out'' a PoC
  \item Introduced a ``providers'' API to compliment boot modules
  \item Created even more code duplication, highlighted need to unify codebases
  \item Abandoned in favor of unification
  \end{itemize}
\end{frame}

\subsection{EFIzation}

\begin{frame}
  \frametitle{Table of Contents}
  \tableofcontents[currentsection]
\end{frame}

\begin{frame}
  \frametitle{UEFI Driver Primer}
  \begin{itemize}
  \item \texttt{EFI\_DRIVER\_BINDING} API allows registration of new drivers
  \item Drivers have probe/attach functions for \texttt{EFI\_HANDLE}s
  \item Can attach various interfaces to an \texttt{EFI\_HANDLE}
  \item Can also create new \texttt{EFI\_HANDLE}s to represent virtual devices
  \item New devices will be automatically probed by all registered drivers
  \item UEFI spec guarantees GPT/MSDOSFS drivers
  \end{itemize}
\end{frame}

\begin{frame}
  \frametitle{EFIzation Effort}
  Observation: UEFI provides an API with the same functionality as
  \texttt{libstand}; use it instead!
  \begin{itemize}
  \item First effort to unify \texttt{boot1} and \texttt{loader.efi}
  \item \texttt{boot1} and \texttt{loader.efi} would use
    \texttt{EFI\_SIMPLE\_FILE\_SYSTEM} interface
  \item Produced ``shim'' drivers: UEFI-to-\texttt{libstand},
    \texttt{libstand}-to-UEFI
  \item \texttt{libstand} drivers sat under
    \texttt{EFI\_SIMPLE\_FILE\_SYSTEM} interface
  \item \texttt{boot1} directly utilized
    \texttt{EFI\_SIMPLE\_FILE\_SYSTEM} to find and load \texttt{loader.efi}
  \item \texttt{loader.efi} continued to use \texttt{libstand}
    interface, which talked through the other shim to UEFI API
  \end{itemize}
\end{frame}

\begin{frame}
  \frametitle{UEFI Drivers}
  \begin{itemize}
  \item GELI was implemented directly as a UEFI driver
  \item GELI used \texttt{EFI\_DRIVER\_BINDING} API to register itself
    as a driver, created new device handles for GELI volumes it detects
  \item Benefit: this carries over across the
    \texttt{boot1}/\texttt{loader.efi} boundary
  \item \texttt{efipart} \emph{mostly} converted to a UEFI driver
  \item Issues with \texttt{bcache} prevented full conversion
  \end{itemize}
\end{frame}

\subsection{Keys and Crypto}

\begin{frame}
  \frametitle{Table of Contents}
  \tableofcontents[currentsection]
\end{frame}


\begin{frame}
  \frametitle{Managing Keys in the Loader}
  \begin{itemize}
  \item UEFI driver interface solves one half of the problem raised by
    the \texttt{boot1}/\texttt{loader.efi} gap
  \item \texttt{EFI\_HANDLE}s registered in \texttt{boot1} are available
    in \texttt{loader.efi}
  \item This provides \emph{access}
  \item Still need \emph{keys} to pass into the kernel
  \end{itemize}
\end{frame}

\begin{frame}
  \frametitle{UEFI KMS Interface}
  \begin{itemize}
  \item UEFI defines a key management system (KMS) interface
  \item Implemented a simple in-memory key database as a driver which
    provides this interface.
  \item GELI driver attempts to locate a KMS during initialization
  \item GELI stores/retrieves keys from its KMS
  \item Kernel metadata step also locates the KMS, transfers all keys
    into the kernel via the \texttt{keybuf} interface
  \end{itemize}
\end{frame}

\begin{frame}
  \frametitle{Kernel Key Intake Buffer (\texttt{keybuf})}
  \begin{itemize}
  \item Provide a better way of getting keys into the kernel
  \item Uses kernel metadata functionality to deliver (by default) up
    to 64 keys, each up to 4096 bits long
  \item Keys have a type code indicating their format
  \item Picked up by \texttt{crypto}, then subsequently available to
    other drivers for initialization
  \item GELI passes in hashed passwords
  \item Designed to be extended to work with hardware crypto
  \end{itemize}
\end{frame}

\begin{frame}
  \frametitle{Boot Crypto Framework (\texttt{boot\_crypto})}
  \begin{itemize}
  \item Inherited code from X86 BIOS implementation, but created a
    separate codebase
  \item X86 BIOS is space-constrained and only supports AES; UEFI is
    not space-constrained
  \item \texttt{boot\_crypto} is designed around a generic algorithm
    interface with pluggable backends
  \item Designed to anticipate overhaul of \texttt{crypto} framework
  \item Also designed to support hardware crypto device implementations
  \end{itemize}
\end{frame}

\begin{frame}
  \frametitle{If/When Trustworthy Hardware KMS/Crypto Exists\ldots}
  Thinking ahead to a time when there is a trustworthy hardware KMS
  implementation was a consideration in this design
  \begin{itemize}
  \item In-memory KMS detection aborts if it detects another KMS
    device (this also deals with \texttt{boot1} and
    \texttt{loader.efi} both having to attempt to register the
    in-memory KMS device)
  \item GELI should ``just work'', as it talks through the KMS and
    \texttt{boot\_crypto} interfaces
  \item \texttt{boot\_crypto} would need to add support
  \item ``Keys'' would likely consist of ID numbers for keys stored in
    KMS
  \item \texttt{keybuf} interface could easily add another key type for key IDs
  \end{itemize}
\end{frame}

\subsection{Second Unification}

\begin{frame}
  \frametitle{Table of Contents}
  \tableofcontents[currentsection]
\end{frame}

\begin{frame}
  \frametitle{Benefits of EFIzed Approach}
  \begin{itemize}
  \item \texttt{boot1} reduced to a very minimal program, uses same
    codebase as \texttt{loader.efi}
  \item Seamless integration with firmware-provided drivers
  \item Dropped MSDOSFS driver
  \item Provided framework for hot-plugging support (\texttt{bcache}
    got in the way of full implementation)
  \item Laid groundwork for exporting driver code to others
  \end{itemize}
\end{frame}

\begin{frame}
  \frametitle{Drawbacks of EFIzed Approach}
  \begin{itemize}
  \item UEFI does a bad job at supporting non-Microsoft systems and
    interfaces
  \item \texttt{EFI\_SIMPLE\_FILE\_SYSTEM} interface is designed
    around MSDOSFS, sits uncomfortably in a VFS interface
  \item Difficult to present the same information in boot shell as in
    current \texttt{loader}
  \item ZFS boot environments lost when talking over UEFI interfaces
  \end{itemize}
  Personal takeaway: started with moderately positive views on UEFI,
  ended with moderately negative views.
\end{frame}

\begin{frame}
  \frametitle{Simplified Unification}

  Eventually, code changes in HEAD broke the patches in non-trivial
  ways, and the drawbacks of EFIzed approach were becoming clear.
  \begin{itemize}
  \item Moved UEFI-to-\texttt{libstand} shim out to an independent
    review (still up for review)
  \item Dropped \texttt{libstand}-to-UEFI shim altogether
  \item Refactored \texttt{boot1} to use \texttt{libstand}
  \item Recovered simplicity, information at boot shell, ZFS boot environments
  \item Casualty: progress towards hot-pluggable devices at boot time
  \end{itemize}
\end{frame}

\begin{frame}
  \frametitle{Refurbishing Efforts}
  \begin{itemize}
  \item \texttt{efipart} had moved away from a static-numbered,
    array-based storage scheme for device handles (right move)
  \item \texttt{efipart} had also split up device handles by drive
    type (also right move)
  \item Found an integer overflow bug in
    \texttt{efipart\_realstrategy} when attempting to read past the
    end of a device (caused crash)
  \item \texttt{efipart} was manually parsing partition tables and
    using base device handles
  \item This didn't work at all with GELI, so had to revert to direct
    access through partition device handles
  \end{itemize}
\end{frame}

\subsection{Testing and Review}

\begin{frame}
  \frametitle{Table of Contents}
  \tableofcontents[currentsection]
\end{frame}

\begin{frame}
  \frametitle{Kernel gets \texttt{keybuf} Interface}
  \begin{itemize}
  \item \texttt{keybuf} patch went into kernel first
  \item X86 BIOS GELI support started using \texttt{keybuf}
  \item Legacy environment variable method still supported, eventually
    to be phased out
  \item \emph{Definitely} the right move to put \texttt{keybuf} in first
  \item Anyone using a recent kernel can use UEFI GELI without a
    kernel update
  \end{itemize}
\end{frame}

\begin{frame}
  \frametitle{Testing on QEMU and Real Hardware}
  \begin{itemize}
  \item QEMU testing setup had a large number of GELI disks, including
    encrypted/unencrypted UFS, ZFS, also an X86 BIOS setup
  \item Tested all combinations on QEMU
  \item I had also been using a root-on-ZFS laptop with its
    L2ARC/Intent log stored on GELI volumes since the EFIzed version
  \item Finally, converted a laptop over to a full GELI root-on-ZFS setup
  \item Works perfectly (except I forgot the \texttt{-R} when taking
    the ZFS snapshots\ldots)
  \end{itemize}
\end{frame}

\section{Future Plans}

\begin{frame}
  \frametitle{Table of Contents}
  \tableofcontents[currentsection]
\end{frame}

\begin{frame}
  \frametitle{Plans for GRUB/Coreboot}
  \begin{itemize}
  \item GRUB is (reportedly) the best way to achieve a Coreboot setup
  \item Coreboot is arguably a better option (where it's supported)
  \item GRUB already supports GELI, but needs to be updated to use the
    \texttt{keybuf} interface
  \item Initial conversations with GRUB developers indicates this
    shouldn't be hard
  \end{itemize}
\end{frame}

\begin{frame}
  \frametitle{My Long-Term Plans}

  My overall agenda can be described as ``OS-level tamper-resilience''
  \begin{itemize}
  \item Full-disk encryption (GELI)
  \item Trust framework and kernel/module signing
  \item Active use of coreboot/setup guides
  \item Secure suspend/resume
  \item Other uses of trust framework
  \end{itemize}
\end{frame}

\end{document}
