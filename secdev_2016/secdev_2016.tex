\documentclass{beamer}
\usepackage{MnSymbol}

\title{Combining Formal Methods and Industrial Pragmatics}
\author{Eric L. McCorkle}

\begin{document}
\begin{frame}
  \titlepage
\end{frame}

\begin{frame}
  \frametitle{Dependent Types (very briefly)}
  Dependently-typed languages are equipped with very powerful type systems
  \begin{itemize}
    \item Type systems are strong enough to express full specifications
    \item Type-checking amounts to proving implementations behave
      according to spec
    \item Must still get the specification right!  (Who watches the
      watchers?)
    \item Proofs resemble code, must be developed and maintained like code
    \item Powerful tool for building security in!
  \end{itemize}
\end{frame}

\begin{frame}
  \frametitle{``Industrial'' Programming Languages}
  What makes a language successful in the ``Real World''?
  \begin{itemize}
    \item Realities: very large codebases, code evolution, staff
      turnover, differing skill-levels, cost/benefit tradeoffs,
      compatibility, etc.
    \item Resist bit-rot, withstand inelegance, hold up under refactoring
    \item ``Harm-reduction'' often works better than ``thou shalt''
  \end{itemize}
  What tends to work well?
  \begin{itemize}
    \item Optimize for least eventual cost (or pain)
    \item Modularization, encourage good practices, code reuse
    \item Present complex ideas in an accessible fashion
    \item API/library design is an art-form to be celebrated
  \end{itemize}
\end{frame}

\begin{frame}
  \frametitle{Vision of ``Industrial'' Dependently-Typed Languages}
  How can we make depndent types and verification suitable for
  industrial programming?
\end{frame}

\begin{frame}
  \frametitle{Gradual Proof Checking (and Typing)}
  \begin{itemize}
    \item Making people prove their entire program correct before running
      it is a non-starter for industry
    \item People develop software iteratively
    \item Cost/benefit tradeoffs, risk profiles, ROI differ over components
    \item ``Two-phase'' type/proof-checking: first phase is decidable,
      second phase does verification.
    \item Prove critical components correct, rely on testing for the
      rest, gradually work your way outward
    \item Provides a smooth transition from prototypes to verified
      systems
    \item Go a step further: do this with type checking in general
      (gradual typing)
  \end{itemize}
\end{frame}

\begin{frame}
  \frametitle{Managing Large Verified Codebases}
  How would we manage large bodies of proofs about code?
  \begin{itemize}
    \item Proofs very closely resemble code
    \item Provide ability to automate proofs using the same language
      as the code
    \item Apply known techniques that work for code management:
      modularization, small units of functionality, API design
    \item Draw on historically successful language concepts (OO
      features, typeclasses, etc) to design constructs for managing
      verification
    \item Draw on (and perhaps refactor) parts of mathematics,
      particularly abstract algebra
  \end{itemize}
\end{frame}

\begin{frame}
  \frametitle{A Vision of Industrial Dependent-Typed Languages}
  \begin{itemize}
    \item View as a specification/reasoning system built into the language
    \item Proof obligations provided as an artifact of compilation,
      usable to other tools)
    \item Gradual ``pay-as-you-go'' typing and proof-checking
    \item Start with no spec, leave proof obligations unproven
    \item Develop spec iteratively, prove obligations where advantageous
    \item Proofs look like code, make use of traditional software
      engineering techniques
    \item A fully-mature module has specs, proofs, and facilities for
      automating proofs about the module
  \end{itemize}
\end{frame}

\begin{frame}
  \begin{itemize}
    \item Happy to discuss these ideas in greater detail
    \item I am actively working on a language to implement these ideas
    \item Particularly interested in how to improve infosec through
      better languages
    \item Email me (eric@metricspace.net) or come find me to talk more
  \end{itemize}
\end{frame}

\end{document}
