\documentclass{beamer}

\title{Design of a Public-Key Trust System for FreeBSD}
\author{Eric L. McCorkle}

\begin{document}

\begin{frame}
  \titlepage{}
\end{frame}

\section{Introduction}

\begin{frame}
  \frametitle{Motivating Example}
  Consider a proposal for signing the kernel and modules:
  \begin{itemize}
  \item Extend Executable Linkable Format (ELF) to carry public-key
    signatures
  \item Sign kernel and modules with a private key for each build
  \item Kernel and boot loader carry the verification (public) key
  \item Loader checks kernel/module signatures before booting
  \item Kernel checks module signatures before allowing them to be loaded
  \item UEFI and GRUB both have equivalent facilities
  \end{itemize}
\end{frame}

\begin{frame}
  \frametitle{Cryptography as Trust}

  Signed kernels and modules are an example of \emph{cryptography as
    trust}.
  \begin{itemize}
  \item Cryptography is most often viewed as a \emph{confidentiality}
    mechanism
  \item However, it can also fulfill other purposes, such as
    \emph{authorization}
  \item In FreeBSD (and many other systems) the kernel
    enforces authorization rules
  \item Relies on memory protection, internal tables, user IDs, etc to
    restrict \emph{who may access/modify}
  \item Signed kernel modules allow authorization to restrict the
    \emph{content} of the modules/kernel
  \end{itemize}
\end{frame}

\begin{frame}
  \frametitle{Public-Key Cryptography in System Context}

  Public-key cryptography can extend and/or strengthen many security
  features of operating systems:
  \begin{itemize}
  \item Signed kernels, modules, executables, libraries
  \item Distribution and delegation in a capabilities-based access
    control system (capsicum)
  \item Strong (cryptographic) data access controls
  \item ``Traditional'' public-key functions (session key negotiation,
    protocols)
  \item System-level trust management
  \end{itemize}
\end{frame}

\begin{frame}
  \frametitle{Trust Management}

  Trust management is vital in a public-key system.
  \begin{itemize}
  \item Some public key (or set of them) serves as a \emph{root of trust}
  \item Trust can be extended to additional keys through signatures
  \item Chains of trust can be formed by signing each successive key
    with the previous key
  \item Public-Key Infrastructure (PKI) systems allow for a tree-like
    structure
  \item Other systems (PGP) use a web-of-trust (general graph)
  \end{itemize}
\end{frame}

\section{Design Overview}

\begin{frame}
  \frametitle{Table of Contents}
  \tableofcontents[currentsection]
\end{frame}

\begin{frame}
  \frametitle{Trust System Design for FreeBSD}
  \begin{itemize}
  \item \emph{Runtime Trust Database}: In-kernel API for managing
    root/intermediate keys
  \item \texttt{devfs}-based interface for adding/revoking
    intermediate keys from userland
  \item \emph{Trust base configuration}: Configurations for building
    in root keys, loading intermediate keys at boot.
  \item Signed ELF binary format extension, conventions for signing
    vital config files
  \item NetBSD VeriExec integration
  \end{itemize}
\end{frame}

\begin{frame}
  \frametitle{Kernel API}
  \begin{itemize}
  \item Root certificates are established at boot, cannot be changed
    during runtime (without hardware intervention)
  \item Database tracks trust relationships, forms a forest with root
    keys as roots
  \item Intermediate certificates can be added, providing they are
    signed by an existing root or intermediate certificate
  \item All keys have a revocation list (initially empty), can be set
    for any key
  \item Any intermediate certificates in their signers' revocation
    lists are removed, along with their descendents
  \item Can check a signature against the database
  \item Can enumerate database
  \end{itemize}
\end{frame}


\begin{frame}
  \frametitle{\texttt{devfs} Interface}
  \begin{itemize}
  \item Present device nodes under \texttt{/dev/trust/}
  \item Control interface at \texttt{/dev/trust/trustctl}:
    \begin{itemize}
    \item Write an X.509 certificate signed by a trusted certificate
      to install as an intermediate certificate (check revocation
      lists)
    \item Write an X.509 revocation list signed by a trusted
      certificate to install it as the signer's revocation list (and
      do revocations)
    \item Use binary DER encoding (easy/safe to parse) for input
    \end{itemize}
  \item Enumeration interfaces at \texttt{/dev/trust/certs},
    \texttt{/dev/trust/rootcerts}:
    \begin{itemize}
    \item \texttt{/dev/trust/certs} reads back all certificates
    \item \texttt{/dev/trust/rootcerts} reads back just roots
    \item Read back certificates in PEM encoding, allows nodes to be
      used as CAcert configuration for many applications
    \end{itemize}
  \item Could also render entire forest as directory structure
  \end{itemize}
\end{frame}

\begin{frame}
  \frametitle{Obtaining Root Keys}
  There are several options for obtaining root keys at startup:
  \begin{itemize}
  \item Build directly into loader/kernel
    \begin{itemize}
    \item Advantage: Secure, better cipher suite
    \item Disadvantage: Inflexible, difficult to recover from mishaps,
      bad for standard images
    \end{itemize}
  \item Obtain from secure boot infrastructure or hardware
    \begin{itemize}
    \item Advantage: Integration with hardware/secure boot, flexible
    \item Disadvantage: Often weak crypto suites (RSA 2048 is as good
      as it gets)
    \end{itemize}
  \item Pass from loader to kernel via \texttt{keybuf}
    \begin{itemize}
    \item Advantage: Flexible, full cipher suite
    \item Disadvantage: Less secure than compiling in
    \end{itemize}
  \end{itemize}
\end{frame}

\begin{frame}
  \frametitle{Trust Base Configuration}
  \begin{itemize}
  \item Establishes trust configuration for builds and system startup
  \item Store trust root certs at \texttt{/etc/trust/root/certs} (keys
    at \texttt{/etc/trust/root/keys} if we have them)
  \item Intermediate trust certs at \texttt{/etc/trust/certs} are
    loaded by \texttt{rc} at boot
  \item Trust root keys converted to C source, compiled into a static library
  \item Ultimately compiled into loader and possibly kernel
  \item Kernels may be signed with an ephemeral intermediate key,
    stored at \texttt{/boot/kernel/cert.pem}
  \end{itemize}
\end{frame}

\begin{frame}
  \frametitle{Example Trust Configurations}
  \begin{itemize}
  \item Preferred configuration is one locally-generated trust root
    key
  \item Third-party vendor certs don't have a corresponding signing key
  \item In the preferred configuration, all vendor keys are signed by
    the local trust root key
  \item Standard distributions can be signed with FreeBSD foundation's
    vendor key
  \item Likely will want to have installer generate the local key,
    then inject it into the loader, then sign FreeBSD's vendor cert
  \item Alternative config for high security networks has no local
    keys, only the network's vendor cert, builds produced and signed
    on a central machine
  \end{itemize}
\end{frame}

\begin{frame}
  \frametitle{Formats and Tooling}
  \begin{itemize}
  \item OpenSSL is part of FreeBSD base system
  \item X509 certificates used by many applications, sensible format
  \item DER binary encoding is best for input format to device nodes
    \begin{itemize}
    \item Easy to parse
    \item Disinguished encoding allows byte-to-byte comparisons
    \item Can be generated by \texttt{openssl} command-line tool
    \end{itemize}
  \item PEM encoding is preferable for outputting trusted keys (used
    by many applications)
  \item DER for input, PEM for output
  \end{itemize}
\end{frame}

\begin{frame}
  \frametitle{Signed Executables}
  \begin{itemize}
  \item ELF file format based on sections, already has conventions for
    extra metadata (DWARF, \texttt{.comment}, \texttt{.note}, etc)
  \item Cryptographic Message Syntax (CMS) supported by OpenSSL/PKI,
    allows for detached signatures
  \item Signed executables have a \texttt{.sign} section, containing a
    CMS detached signature
  \item Signatures are computed with a same-sized, zeroed-out
    \texttt{.sign} section
  \item Signatures in this scheme can be added/verified/removed using
    \texttt{objcopy} and \texttt{openssl}
  \end{itemize}
\end{frame}

\begin{frame}
  \frametitle{A Note on Alternatives}

  Several altenative approaches exist:
  \begin{itemize}
  \item GRUB uses detached GPG signatures
  \item Linux has a system call-based kernel keyring feature
  \end{itemize}

  Reasons for not going with the alternatives:
  \begin{itemize}
  \item Signed ELF binary scheme is compatible with existing
    tools/installers; detached GPG signatures aren't
  \item \texttt{devfs} control interface can be used by existing
    applications w/o modification
  \item PGP-compatible tools not in FreeBSD base system
  \item Web-of-trust is arguably the wrong model for such a system
  \item Revocation in PGP systems done by the key \emph{owner}, not
    the signitories
  \end{itemize}
\end{frame}

\begin{frame}
  \frametitle{NetBSD VeriExec Framework}

  The NetBSD VeriExec framework also provides a file integrity
  checking mechanism
  \begin{itemize}
  \item MAC registry specifies authentication codes for arbitrary files
  \item MACs are checked upon loading files, \texttt{execve} calls, etc.
  \item Advantage: out-of-band integrity checks (doesn't require
    in-file signatures like signed ELF)
  \item Cannot manage \emph{delegated} trust, less flexible than a
    public-key mechanism
  \item Basic integration: allow MAC registries to be loaded at any
    point, if signed by a trusted key
  \end{itemize}
\end{frame}

\begin{frame}
  \frametitle{UUID-Marked Executables}
  \begin{itemize}
  \item VeriExec associates MACs with a path; can be inflexible
  \item Signed ELFs can be moved around freely (advantage of in-file
    metadata)
  \item Hybrid mechanism: add a UUID to each ELF, can be generated
    with 128-bit hash (SHA-1, RipeMD-128)
  \item Allow VeriExec to associate MACs to UUIDs as well as paths
  \item Executables can be marked with UUIDs once, never need to be
    modified to add additional signatures
  \item UUID-marked executables can have other administrative uses
  \end{itemize}
\end{frame}

\section{Post-Quantum Cryptography}

\begin{frame}
  \frametitle{Table of Contents}
  \tableofcontents[currentsection]
\end{frame}

\begin{frame}
  \frametitle{The Quantum Machines are Coming!}
  \begin{itemize}
  \item Decent estimate: quantum machines capable of attacking
    existing public-key crypto likely to arrive some time between 5
    and 50 years from now
  \item Hidden-subgroup attack breaks RSA, elliptic-curve/classical
    discrete logs (all common public-key crypto)
  \item Grover iteration: quadratic-speed attack against
    symmetric-key, MACs, hashes (halfs bit security)
  \item Grover iteration is a theoretical attack (requires large
    quantum memory, very long stability)
  \item Short version: symmetric-key, hashes, MACs safe, public-key
    exchange/signatures broken
  \end{itemize}
\end{frame}

\begin{frame}
  \frametitle{Post-Quantum Cryptography}
  \begin{itemize}
    \item \emph{Post-quantum} cryptography aims to develop
      \emph{classical} cryptographic methods that are secure against
      quantum attacks (distinct from quantum cryptography)
    \item Viable post-quantum key exchange being deployed (SIDH)
    \item Post-quantum signatures don't have as nice a picture
    \item Hash-based signatures: reliable, very mature (date back to
      Lamport) but have serious caveats
    \item Other post-quantum signature schemes are still under active
      research, too new, or extremely impractical ($>1$Mib signatures,
      etc)
  \end{itemize}
\end{frame}

\begin{frame}
  \frametitle{Hash-Based Signatures}
  \begin{itemize}
  \item XMSS: Stateful hash-based signatures
    \begin{itemize}
    \item Good for finite number of signatures
    \item Signature size varies, but reasonable parameters give 1-4Kib
    \item Non-standard interface: updates ``state'' on every signing
      operation
    \item Re-signing with old states destroys security properties
    \item Adam Langley: ``Giant foot-cannon''
    \end{itemize}
  \item SPHINCS: Stateless (big) hash-based signatures
    \begin{itemize}
    \item Classic public-key signature interface, no state
    \item Signature size is 40Kib
    \end{itemize}
  \end{itemize}
\end{frame}

\begin{frame}
  \frametitle{Using Hash-Based Signatures in Trust}

  The trust framework provides use cases where both schemes can be
  used practically:
  \begin{itemize}
    \item Stateful signatures are ideal for batch-signing: create
      key-pair, sign, destroy private key
    \item Stateful signatures also good for non-persisted key used,
      controlled by kernel, generated at boot and destroyed at
      shutdown.
    \item Could use stateful signatures to issue delegated
      credentials valid only for system uptime
    \item SPHINCS signatures good for signing big messages, or signing
      relatively small numbers of messages
    \item Ideal for VeriExec manifests (likely to be much larger than
      40Kib)
  \end{itemize}
\end{frame}

\section{Applications and Implementation}

\begin{frame}
  \frametitle{Table of Contents}
  \tableofcontents[currentsection]
\end{frame}

\begin{frame}
  \frametitle{Applications}
  \begin{itemize}
  \item Signed kernel and modules
  \item Trusted boot
  \item Signed executables/configuration files
  \item System-wide certificate configurations
  \item Delegation of capabilities to remote systems
  \end{itemize}
\end{frame}

\begin{frame}
  \frametitle{Implementation Roadmap}
  \begin{itemize}
  \item Crypto library for kernel/loader (crypto overhaul is an open
    topic)
  \item In-kernel runtime trust database, \texttt{devfs} interface
  \item Modify loader to check signatures
  \item Add code to check kernel module signatures
  \item Implement \texttt{signelf} (done)
  \item Modify build system to produce static library containing root
    keys from trust base config, sign executables
  \item Modify \texttt{rc} to load intermediate certificates at boot
  \end{itemize}
\end{frame}

\begin{frame}
  \frametitle{Crypto Overhaul (brief)}
  \begin{itemize}
  \item Kernel crypto, OpenCrypto generally in need of overhaul (old,
    poor organization)
  \item No public-key, no PKI parsing
  \item Loader only has a stop-gap measure for implementing GELI
  \item Popular options:
    \begin{itemize}
    \item Import OpenSSL (tried once, failed)
    \item LibreSSL (developed by OpenBSD)
    \item BearSSL (new, still under development)
    \item Earlier versions of this proposal included minimal PKI library
    \end{itemize}
  \item Any solution will need to add new ciphers (use FreeBSD OID
    space to create new OIDs, upstream to crypto)
  \end{itemize}
\end{frame}

\begin{frame}
  \frametitle{Kernel Key Database, \texttt{devfs}}
  \begin{itemize}
  \item Basic forest data structure with public keys/revocation lists
  \item Hardware interface: likely have abstraction layer for storing
    individual keys
  \item Maintain forest structure in kernel
  \item \texttt{devfs} interface ends up being a straightforward use
    of kernel API
  \item Kernel/Loader then has API for checking public-key signatures
    (main goal)
  \item Use this to check signatures on executables, files, etc.
  \end{itemize}
\end{frame}

\begin{frame}
  \frametitle{The \texttt{signelf} Utility}
  \begin{itemize}
  \item Batch signer, streamlined tool for signing large numbers of
    ELF binaries
  \item More convenient than using \texttt{objcopy}/\texttt{openssl}
  \item Gets keys/certs from system trust configuration by default
  \item Can generate an ephemeral key-pair for signing
  \item Writes out verification key for ephemeral key-pair, destroys
    signing key
  \item Initial implementation using OpenSSL complete
  \end{itemize}
\end{frame}

\begin{frame}
  \frametitle{Build System, \texttt{rc} Modifications}
  \begin{itemize}
  \item Convert trust root certificates into C code early in build
  \item Create static library (\texttt{librootkeys.a})
  \item Loader and Kernel can then compile in keys
  \item \texttt{rc.d} script to install intermediate
    certificates/revocation lists via \texttt{devfs} interface
  \end{itemize}
\end{frame}

\end{document}
